%% =======================================================
\section{Beispiel Relationen}
%% ---------------------------------
\begin{frame}[fragile]
  \frametitle{Beispiel: Aufgabe zu Relationen}
  { \small
\begin{verbatim}
Die folgende Aufgabe bezieht sich auf Relationen
über dem Grundbereich mkSet [ 1 , 2 , 3 , 4 , 5 ]

Gesucht ist ein Ausdruck (Term) mit dieser Bedeutung:
    {(1 , 1) , (2 , 5)}
Der Ausdruck soll höchstens die Größe 15 haben.
Sie dürfen diese Symbole benutzen
    zweistellige : [ * , + , - ]
    einstellige  : [ ]
    nullstellige : [ ]
und diese vordefinierten Konstanten:
    R = {(1 , 4) , (1 , 5) , (2 , 3) , (3 , 1) , (4 , 5)}
    S = {(1 , 4) , (3 , 3) , (4 , 2) , (4 , 4) , (5 , 5)}
\end{verbatim}
  }  
\end{frame}
%% ---------------------------------
\begin{frame}[fragile]
  \frametitle{Lösung}
  {\small
    eingesendete Lösung: \verb|R * S * R * R|
\begin{verbatim}
Der Wert Ihres Terms ist die Relation
    {(1 , 1) , (2 , 4) , (2 , 5)}
stimmen die Relationen überein?
Stimmen die Menge
Aufgabenstellung = mkSet [ ( 1 , 1 ) , ( 2 , 5 ) ]
und die Menge
Einsendung = mkSet [ ( 1 , 1 ) , ( 2 , 4 ) , ( 2 , 5 ) ]
überein?
Ist die Menge
Aufgabenstellung = mkSet [ ( 1 , 1 ) , ( 2 , 5 ) ]
Teilmenge der Menge
Einsendung = mkSet [ ( 1 , 1 ) , ( 2 , 4 ) , ( 2 , 5 ) ] ?
  Ja.
Ist die Menge
Einsendung = mkSet [ ( 1 , 1 ) , ( 2 , 4 ) , ( 2 , 5 ) ]
Teilmenge der Menge
Aufgabenstellung = mkSet [ ( 1 , 1 ) , ( 2 , 5 ) ] ?
  Nein, diese Elemente sind in Einsendung, 
aber nicht in Aufgabenstellung mkSet [ ( 2 , 4 ) ]
\end{verbatim}
    % \pause\vfill 

    % korrekte Einsendung \verb|( R * S * R * R ) - ( R * R * S )|
    % \begin{verbatim}
    % Der Wert Ihres Terms ist die Relation
    % {(1 , 1) , (2 , 5)}

    % stimmen die Relationen überein?
    % Ist die Menge
    % Aufgabenstellung = mkSet [ ( 1 , 1 ) , ( 2 , 5 ) ]
    % Teilmenge der Menge
    % Einsendung = mkSet [ ( 1 , 1 ) , ( 2 , 5 ) ]
    % ?

    % Ja.

    % Ist die Menge
    % Einsendung = mkSet [ ( 1 , 1 ) , ( 2 , 5 ) ]
    % Teilmenge der Menge
    % Aufgabenstellung = mkSet [ ( 1 , 1 ) , ( 2 , 5 ) ]
    % ?

    % Ja.

    % Bewertung der Einsendung: Okay
    % \end{verbatim}
  }
\end{frame}
%% =======================================================
\section{Beispiel Mengen}
%% ---------------------------------
\begin{frame}[fragile]
  \frametitle{Beispiel: Aufgabe zu Mengen}
  {\small
\begin{verbatim}
Gesucht ist ein Ausdruck (Term) mit dieser Bedeutung:
    {2, {}, {3, {4}}}
Der Ausdruck soll höchstens die Größe 40 haben.
Sie dürfen diese Symbole benutzen
    zweistellige : [ + , - , & ]
    einstellige  : [ pow ]
    nullstellige : [ ]
und diese vordefinierten Konstanten:
    A = {1, 2}
    B = {2, 3}
    C = {3, 4}
\end{verbatim}
  }
\end{frame}
%% =======================================================
\section{Beispiel Zahlensysteme}
%% ---------------------------------
\begin{frame}[fragile]
  \frametitle{Beispiel: Aufgabe zu Zahlensystemen}
  {\small
    autotool-Aufgabe: 
\begin{verbatim}
Stellen Sie
    Zahl
        { basis = 5 , ziffern = [ 3 , 1 , 4 ] }
in der Basis 7 dar.
\end{verbatim}
    \pause\vfill

    % Lösung:
    % \begin{verbatim}
    % Zahl
    % { basis = 7 , ziffern = [ 1 , 5 , 0 ] }
    % \end{verbatim}

    autotool-Antwort:
\begin{verbatim}
gelesen: Zahl
             { basis = 7 , ziffern = [ 1 , 5 , 0 ] }
partiell korrekt?
die Basis soll 7 sein.
    Ja.
jede Ziffer soll in [ 0, 1 .. basis - 1 ] liegen.
total korrekt?
Stimmen die Bedeutungen der Zahlen überein?
    Ja.
Bewertung der Einsendung: Okay
\end{verbatim}
  }
\end{frame}
%% =======================================================
\section{Beispiel Multiplikative Inverse}
%% ---------------------------------
\begin{frame}[fragile]
  \frametitle{Beispiel: Rechnen mit Restklassen}
  {\small
    autotool-Aufgabe: 
\begin{verbatim}
Berechnen Sie das multiplikative Inverse von
    7
modulo
    11
falls es existiert!
\end{verbatim}
    \pause\vfill

    % Lösung:
    % \begin{verbatim}
    % Just 8
    % \end{verbatim}

    autotool-Antwort:
\begin{verbatim}
gelesen: Just 8
partiell korrekt?
Ja.
total korrekt?
Ist 7 * 8 = 56 = 1 mod 11 ?
Ja, denn 56 = 5 * 11 + 1
Bewertung der Einsendung: Okay
\end{verbatim}
  }
\end{frame}
%% =======================================================
